\documentclass[a4paper,10pt,uplatex]{jsarticle}


% 数式
\usepackage{amsmath,amssymb,amsthm,bm}
\usepackage{mathtools}
\mathtoolsset{showonlyrefs}
\usepackage{physics}
\usepackage{siunitx}
\usepackage[thicklines]{cancel}
\usepackage{comment}
\usepackage{graphicx}
\usepackage[dvipdfmx]{color}
\usepackage{here}
\usepackage{tikz}
\usepackage{enumitem}
% \setlist{nolistsep}

\newcommand{\Set}[1]{\{#1\}}
\newcommand{\miss}{\textcolor{red}{\#}}
\newcommand{\N}{\mathbb{N}}
\newcommand{\R}{\mathbb{R}}
\newcommand{\C}{\mathbb{C}}
\newcommand{\Z}{\mathbb{Z}}
\newcommand{\Q}{\mathbb{Q}}
\newcommand{\T}{\mathbb{T}}
\newcommand{\inv}[1]{{(#1)}^{-1}}
\newcommand{\Mod}{\text{mod}\;}
\newcommand{\End}{\text{End}}
\newcommand{\Ker}{\text{Ker}\;}
\newcommand{\Aut}{\text{Aut}}

\numberwithin{equation}{section}

\newtheoremstyle{mystyle}%
    {5pt}%
    {5pt}%
    {}%
    {\parindent}%
    {\bfseries}%
    {:}%
    { }%
    {}%
\theoremstyle{mystyle}
\newtheorem{dfn}{定義}[section]
\newtheorem{thm}{定理}[section]
\newtheorem{lem}[thm]{補題}
\newtheorem{cor}[thm]{系}
\renewcommand{\proofname}{\textbf{証明}}

% 数式の上下のスペースの変更
\AtBeginDocument{
  \abovedisplayskip     =0.5\abovedisplayskip
  \abovedisplayshortskip=0.5\abovedisplayshortskip
  \belowdisplayskip     =0.5\belowdisplayskip
  \belowdisplayshortskip=0.5\belowdisplayshortskip
}

\begin{document}

\title{松坂「代数系入門」}
\author{}
\date{\today}
\maketitle

主に問題の解答.価値があると思った補足も少し.

\section{整数}
\section{群}
\subsection{写像}
略.
\subsection{群とその例}
\begin{itemize}
    \item[2.] まず右逆元も存在して,それが左逆元と一致することを示す.$ba = e$,$cb=e$とすると,
    \[ ab = eab = cbab = ceb = cb = e \]
    従って
    \[ \forall a \in G, \exists b \in G, ab = ba = e \]
    がわかる.次に左単位元が右単位元でもあることを示す.
    \[ ae = aba = ea = a \]
    これで単位元,逆元の存在が言えたので$G$は群である.
    
    \item[3.] まず単位元の存在を示す.ある元$a_0 \in G$について,$a_0 x = a_0$とする.このとき$\forall a \in G$について,
    \[ \exists v \in G, a = va_0 = va_0x = ax \]
    がわかる.$y a_0 = a_0$についても同様で,これより
    \[ \forall a \in G, a = ax = ya \]
    特に$a = x,y$としたときに$x = y$がわかる.これは単位元なので$e$と表す.次に逆元の存在を示す.
    \[ \exists x,y \in G, ax = ya = e \]
    より,
    \[ y = ye = yax = ex = x \]
    従って逆元が一意に定まる.これより$G$は群である.

    \item[4.] $a \in G$を固定して得られる$G$から$G$への写像$x \mapsto ax$,$x \mapsto xa$は条件より単射である.$G$は有限集合なのでこれらは全射でもある.従って,前問の結果より$G$は群である.
    
    \item[5.] $a$を固定して得られる写像は単射だが,$G$が無限集合の場合全射とは限らなくなる.例えば$\Z$に乗法を与えたものがある(この場合逆元が存在しないことがある).
    
    \item[6.] $o(G) = n$とすると,$\{e, a, a^2, \cdots\, a^n\}$の各要素はいずれも$G$の元だが,どこかに被りがある.$a^k = a^l(k < l)$とすると,$a^{l-k} = e$.したがってある$m \in \N$で$a^m = e$が成り立つ.
    
    \item[7.] $a = a^{-1}$より,$ab = a^{-1}b^{-1} = \inv{ba} = ba$.
    
    \item[8.] (a)は$abab = a^2b^2$より$ba = ab$.(b)は
    \begin{align}
        (ab)^{n+2} &= a^{n+2}b^{n+2} \\
        a(ba)^{n+1}b &= a(a^{n+1}b^{n+1})b \\
        (ba)^{n+1} &= a^{n+1}b^{n+1} = (ab)^{n+1} = (ab)^n ab
    \end{align}
    ここで左辺は
    \begin{equation}
        (ba)^{n+1} = a^{-1}abab \cdots aba = a^{-1}(ab)^{n+1}a = a^{n}b^{n+1}a = (ab)^n ba
    \end{equation}
    より,
    \[ (ab)^n ab = (ab)^n ba \quad \therefore ab = ba \]

    \item [9.] 明らかに$\triangle$は可換で,結合的.空集合$\emptyset$が単位元,逆元は$A$自身として,$P(S)$は可換群をなす.
\end{itemize}

\subsection{部分群と生成系}
\begin{itemize}
    \item[1.] 明らか.
    \item[2.] $m$と$n$の最小公倍数を$l$として$l\Z$.
    \item[3.] ある$a \in H \subset G$を固定すれば,$H \to H$の写像$x \mapsto ax$は全単射.簡約律が成り立つことと単射であることは同値なので,2節問題4の結果が使えて,$H<G$が成り立つ.
    \item[4.] 置換であること($\R \to \R$の全単射であること)は明らか.この形の置換全体は$S(X)$の部分群をなすことは,
    \begin{equation}
        \sigma_{a,b}(\sigma_{a',b'}(x)) = a(a'x+b') + b = aa'x + (ab' + b)
    \end{equation}
    と,単位元が$\sigma_{1,0}$,$\sigma_{a,b}$の逆元が$\sigma_{1/a,-b/a}$で与えられることからわかる.
    \item[5.] $G$が有限群だから,$\forall x \in S$について,ある$n \in \N$で$x^n = e, x^{n-1} = x^{-1}$となる.したがって$S^{-1}$の任意の元は$S$の元の有限個の積で表されるから,$S$によって生成される$G$も$S$の元の有限個の積で表される.
    \item[6.] $S^{-1}$の元と$S'^{-1}$の元も可換であるから,$S \cup S^{-1}$の元の積で表される$H$と$S' \cup S'^{-1}$の元の積で表される$H'$の任意の元も可換である.
    \item[7.] 明らか.
    \item[8.] 2次元平面上で,$x$軸と$y$軸に関する鏡映操作
    \begin{equation}
        \sigma = \mqty[-1 & 0\\0 & 1], \quad \tau = \mqty[1 & 0 \\ 0 & -1], \quad e = \mqty[1 & 0 \\ 0 & 1]
    \end{equation}
    は条件を満たす.このとき自明でない部分群は
    \begin{equation}
        \{e, \sigma\}, \{e, \tau\}, \{e, \sigma\tau\}
    \end{equation}
    \item[9.] $\sigma$を正$n$角形の$2\pi/n$回転,$\tau$をある対称軸に関する鏡映操作とすると,これは条件を満たす.別の対称軸に関する鏡映操作$\tau'$は,図形的な考察により$\tau' = \sigma^{2m} \tau$のように表せる.したがってこの正$n$角形のシンメトリー全体からなる群は$\sigma$と$\tau$により生成され,すべての元は$\sigma^i \tau^j$と表される.
    \begin{equation}
        \sigma = R(2\pi/n),\quad \tau = \mqty[1 & 0\\0 & -1]
    \end{equation}
    と行列で表すとわかりやすい.
    \item[10.] $o(D_4) = 8$だから,真部分群の位数の候補は2,4.位数が2の部分群は$\{e,\tau\}$,$\{e,\sigma^2\}$,$\{e,\sigma\tau\}$,$\{e,\sigma^2 \tau\}$,$\{e,\sigma^3 \tau\}$.位数が4の部分群は$\{e,\sigma,\sigma^2,\sigma^3\}$,$\{e,\tau,\sigma^2,\sigma^2\tau\}$,$\{e,\sigma\tau,\sigma^2,\sigma^3\tau\}$.
    \item[11.] \miss(大変そうなため)
    \item[12.] 四元数.
    \begin{equation}
        i = \mqty[i & 0 \\ 0 & -i],\quad j = \mqty[0 & 1 \\ -1 & 0],\quad k = \mqty[0 & i \\ i & 0],\quad e = \mqty[\imat{2}], \quad m = -e
    \end{equation}
    とすれば,与えられた関係式を全て満たす.このとき
    \begin{equation}
        ji = -k,\quad kj = -i,\quad ik = -j
    \end{equation}
    である.$e,m$は全ての元と可換なので,$e,m$をそれぞれ$1,-1$で表すと,$i,j,k$の逆元は$-i,-j,-k$になる.
    真部分群は$\{\pm 1\}$,$\{\pm 1, \pm i\}$,$\{\pm 1, \pm j\}$,$\{\pm 1, \pm k\}$.
\end{itemize}

\subsection{剰余類分解}
\begin{itemize}
    \item[1.] $Q_l \to Q_r$の写像$aH \mapsto Ha^{-1}$のwell-defined性を確認する.
    \begin{equation}
        aH = bH \Leftrightarrow \forall h \in H, \exists h' \in H, ah = bh' \Leftrightarrow \forall h \in H, \exists h' \in H, ha^{-1} = h'b^{-1} \Leftrightarrow Ha^{-1} = Hb^{-1}
    \end{equation}
    全単射であることは明らか.

    \item[2.] $G$の$H$に関する左剰余類を$\{a_1H,\cdots,a_rH\}$,$H$の$K$に関する左剰余類を$\{b_1K,\cdots,b_sK\}$とする.このとき任意の$x \in G$は,まずある$a_i(i=1,\cdots,r)$と$h \in H$によって$x = a_ih$と書かれる.次に$h$はある$b_j(j=1,\cdots,s)$と$k \in K$によって$h = b_jk$と書かれる.従って$x = a_ib_jk$.これより$G$の$K$に関する左剰余類は$\{a_ib_jK\}(1 \leq i \leq r,1 \leq j \leq s)$である.
    \begin{figure}[h]
        \centering
        \caption{2-4-2}
        \includegraphics[keepaspectratio,width=50mm]{"2-4-2.jpeg"}
    \end{figure}

    \item[3.] 任意の$a,b \in H$について,$a^{-1}b \in H \cap K \subset K$より,$a \equiv b (\Mod H \cap K) \Rightarrow a \equiv b (\Mod K)$.\miss また,$a \not\equiv b(\Mod H\cap K)$のとき$a^{-1}b \notin H \cap K$だが,$a,b\in H$なので,$a^{-1}b \notin K$,つまり$a \not\equiv b(\Mod K)$.従って$H$の$H \cap K$による異なる剰余類は必ず$G$の$K$による異なる剰余類の中に含まれるから,$(H:H \cap K) \leq (G:K)$.
    \begin{figure}[h]
        \centering
        \caption{2-4-3}
        \includegraphics[keepaspectratio,width=50mm]{"2-4-3.jpeg"}
    \end{figure}

    \item[4.] 2,3問で得られた結果を用いていく.$H^{(n)} = H_1 \cap \cdots \cap H_n$とすると,$(G:H^{(n)}) = (G:H_n)(H_n:H^{(n)}) \leq (G:H_n)(G:H^{(n-1)})$.これを繰り返し適用すれば得られる.
\end{itemize}

\subsection{正規部分群と商群}
\begin{itemize}
    \item[1.] 明らか.
    \item[2.] $HK$の任意の元は$hk(h \in H, k \in K)$と表されるが,これの逆元は$k^{-1}h^{-1} \in KH$である.したがって,$HK$が部分群になることの必要十分条件は$HK = KH$になることである.
    \item[3.] \miss
    \item[4.] 前問を使えばすぐわかる.
    \item[5.] $aHa^{-1}$の任意の元は$\sigma_a(x)$(共役)の形に表される.したがって$\sigma_a(x)\sigma_a(y)=\sigma_a(xy) \in aHa^{-1}$,$\sigma_a(e) = e$,$\sigma_a(x)^{-1} = \sigma_a(x^{-1})$より$aHa^{-1}$は部分群.
    \item[6.] $N$が正規だから$HN = NH$である.したがって問題2から$HN < G$.また$\sigma_a(hn) = \sigma_a(h)\sigma_a(n)$だから,$H$が正規ならば$HN$も正規になる.
    \item[7.] 明らか.
    \item[8.] 左剰余類と右剰余類の数はどちらも2で,そのうちの一つは$N$なので,任意の$a \notin N$をとれば$aN = Na \neq N$.したがって$N$は正規.
    \item[9.] 任意の$x \in N$について,$a \sim a$を$e \sim x$の左辺からかけて$a \sim ax$.$a^{-1} \sim a^{-1}$を右辺からかけて$e \sim axa^{-1}$より,$axa^{-1} \in N$.したがって$N$は正規.このとき$a \sim b \Leftrightarrow a^{-1}b \sim e \Leftrightarrow a^{-1}b \in N$より,$a \sim b \Leftrightarrow a \equiv b (\Mod N)$.
    \item[10.] 四元数群がその例.
    \item[11.] $a,b \notin H$をとる.$(aH)(bH) = cH$とすると,任意の$h,h' \in H$についてある$h'' \in H$が存在し,$ahbh' = ch''$.ここで$h = h' = e$ととると$ab = ch''$より$c^{-1}ab \in H$.つまり$ab \equiv c(\Mod H)$なので,$(aH)(bH) = abH$.したがって$ahbh' = abh''$で,これを整理すると,任意の$h \in H$について$hb = bh'$をみたす$h' \in H$が存在することがわかる.つまり$bH = Hb$.これより$H$は正規である.
    \item[14.] 任意の$x \in N_1, y \in N_2$の交換子は,$[x,y] = xyx^{-1}y^{-1} = (xyx^{-1})y^{-1} = x(yx^{-1}y^{-1})$より,$[x,y] \in N_1 \cap N_2 = \Set{e}$.したがって$xy = yx$.
\end{itemize}

\subsection{準同型写像}
\begin{itemize}
    \item[1.] 全射であることに注意するとできる.
    \item[2.] 単位元が$\sigma_{1,0}$,$\sigma_{a,b}\sigma_{c,d}=\sigma_{ac,ad+b}$より,$\sigma_{a,b}^{-1} = \sigma_{1/a,-b/a}$であることに注意すると確認できる.
    \item[3.] $f:\C^* \to \T$を$z \mapsto z/|z|$で定義すると,全射準同型になっている.$\Ker f = \R^+$だから,準同型定理より$\C^* / \R^+ \simeq \T$.
    \item[4.] $f:X \to X'$(全単射)として,$\phi:S(X) \to S(X')$を$\varphi \mapsto f \circ \varphi \circ f^{-1}$で定義する.これは同型写像.
    \item[5.] \miss
    \item[6.] \miss $g:G/N \to G'$を$aN \mapsto f(a)$と定義する.$aN = bN \Leftrightarrow a^{-1}b \in N \Rightarrow a^{-1}b \in N_0 \Leftrightarrow a \equiv b(\Mod N_0) \Rightarrow g(a) = g(b)$より,well-definedである.準同型になっていることもすぐ確かめられる.これより$f = g \circ \varphi$なる$g$が存在する.一意性は,$g':G/N \to G'$で,ある$aN \in G/N$について$g(aN) \neq g'(aN)$とすると,$g \circ \varphi(a) \neq f(a)$となって矛盾.したがって$g = g'$.
    \item[7] まず$G'$が可換群だから$f(aba^{-1}b^{-1}) = f(a)f(b)f(a^{-1})f(b^{-1}) = e'$である.つまり$D \subset \Ker f$.したがって前問より$f = g \circ \varphi$なる準同型$g$が一意的に存在する.
    \item[8.] 
\end{itemize}

\subsection{自己同型写像,共役類}
\begin{itemize}
    \item[1.] 任意の$x,y \in G$について$(xy)^{-1} = x^{-1}y^{-1} = (yx)^{-1}$より,$G$は可換.
    \item[2.] $\sigma_a\sigma_b = \sigma_{ab}$,単位元は$\sigma_e$,$\sigma_a$の逆元は$\sigma_{a^{-1}}$で部分群になる.任意の$f \in \Aut(G)$について$f\sigma_af^{-1}(x) = f(af^{-1}(x)a^{-1}) = f(a)xf(a)^{-1} = \sigma_{f(a)}(x)$より正規.
    \item[3.] 内部自己同型群を$H$として$\varphi:G \to H$を$a \mapsto \sigma_a$とすると,これは準同型.$\sigma_a = I_G \Leftrightarrow \forall x \in G, \sigma_a(x) = axa^{-1} = x$より$ax = xa$.つまり$a \in Z$.したがって$\Ker \varphi = Z$で,準同型定理より$G/Z \simeq H$.
    \item[4.] $f \in \Aut(\Z)$は$f(n) = nf(1)$をみたす.$f(1) = m$とすると$f(n) = mn$.これが全単射になるには$m = \pm 1$でなければならない.したがって$\Aut(\Z) = \{I_\Z, -I_\Z\}$.
    \item[5.] $f \in \Aut(\Q)$は$a,b \in \Z$について$f(a/b) = af(1/b)$を
    みたすから,任意の$n \in \Z$について$f(1/n)$を求めればよい.$f(1+1/n) = f(1) + f(1/n) = (n+1)f(1/n)$より,$f(1/n) = f(1)/n$.したがって$f(1) = \alpha$とすれば$f(1/n) = \alpha/n$と表される.これより$\Aut(\Q) = \{(x \mapsto \alpha x)| \alpha \in \Q\}$.
    \item[6.] $aSa^{-1} = bSb^{-1} \Leftrightarrow Sa^{-1}b = a^{-1}bS \Leftrightarrow a^{-1}b \in N(S) \Leftrightarrow a \equiv b(\Mod N(S))$.最後は左合同.したがって$S$に共役な$G$の異なる部分集合の個数は$(G:N(S))$.
    \item[7.] $H$の共役部分群は位数が$o(H)$と変わらないから,仮定より任意の$a \in G$について$\sigma_a(H) = H$.したがって$H$は正規.
    \item[8.] 任意の$x,y \in N$は任意の$a \in G$とある$h,h' \in H$によって$x = \sigma_a(h),y = \sigma_a(h')$と表されるから,$xy = \sigma_a(h)\sigma_a(h') = \sigma_a(hh')$.当然$e \in N$かつ任意の$a,b \in G$について$\sigma_a(h) = \sigma_b(h') \Leftrightarrow \sigma_a(h^{-1}) = \sigma_b(h'^{-1})$より,$\sigma_a(h) \in N \Leftrightarrow \sigma_a(h)^{-1} = \sigma_a(h^{-1}) \in N$.これで$N$は部分群.次に$N$が正規であることは,$N = \bigcap_{a \in G} \sigma_a(H)$から$\sigma_b(N) = \bigcap_{a \in G} \sigma_{ba}(H)$となるが,$ba$は$a$が動けば$G$全体を尽くすから,$\sigma_b(N) = N$より示される.$H$に含まれる任意の$G$の正規部分群を$S$とすると$S = \bigcap_{a \in G} \sigma_a(S)$で,各$a \in G$について$\sigma_a(S) \subset \sigma_a(H)$だから$S \subset N$.
\end{itemize}

\subsection{巡回群}
\begin{itemize}
    \item[1.] $a$を生成元とすると$f(a^k) = f(a)^k$より準同型像は$f(a)$を生成元とする巡回群.
    \item[2.] $a^{n/d}$を生成元とする巡回群.
    \item[3.] $(k,n)=d$,$m=n/d$とすると$mk$は$n$の倍数になるから,$a^{mk}=e$より$a^k$によって生成される部分群の位数は$m$になる.したがって$a^k$が$G$の生成元になるのは$m=n$,つまり$(k,n)=1$のときで,その逆も然り.
    \item[4.] $a=e$または$b=e$のときや,$a=b$のときは明らかなので,$a \neq b, a \neq e, b \neq e$を考える.$o(ab)=n$とすると,$(ab)^n = e$.$(ba)^n = a^{-1}(ab)^{n+1}b^{-1} = a^{-1}abb^{-1}=e$より,$o(ba) \leq n$.ここで$(ab)^m \neq e(1 \leq m \leq n-1)$より,$(ba)^m = a^{-1}(ab)^{m+1}b^{-1} = e$とすると$(ab)^{m+1} = ab$,つまり$(ab)^m = e$となって矛盾.したがって$o(ba) = o(ab)$.
    \item[5.] $k,l$を自然数とし,$a^k = b^l$とすると,$e = a^{mk} = b^{ml}$だが,$n|ml$となるので,$(m,n)=1$より$n|l$.このとき$b^l=e$だから$a^k=e$.つまり$n|k$.これより$a^k = b^l$をみたす最小の$k,l$は$n,m$.したがって$(ab)^k = a^kb^k = e$をみたす$k$は$a^k=b^k=e$をみたし,このうち最小なものは$mn$となる.
    \item[6.] 
    \item[7.] 部分群の位数は$1$か$p$なので真部分群は持たない.したがって$G$はある$a \in G$によって生成される群そのもので,それは$a$の巡回群である.
    \item[8.] 可換群の部分群は常に正規なので,ここでは単純群を真部分群を持たない群とする.このとき任意の$a \in G(a \neq e)$は$G$の生成元になるので,$G$は巡回群.一つ$a \in G$を固定して$o(G)=n$とすると,どの$k=1,2,\cdots,n-1$についても$a^k$が$G$の生成元になる必要があるが,これは問題3より$(k,n)=1$と同値である.したがって$n$は素数.
    \item[9.] 
\end{itemize}
\subsection{置換群}
\begin{itemize}
    \item[1.] $S_3$の部分群は$\Set{e},\Set{e,(1\;2)},\Set{e,(1\;3)},\Set{e,(2\;3)},\Set{e,(1\;2\;3),(1\;3\;2)},S_3$の6つで,このうち正規なのは$\Set{e},\Set{e,(1\;2\;3),(1\;3\;2)},S_3$である.(一般論として$(G:H)=2$ならば$H$は正規)
    \item[2.] (a)$r$.(b)$\sigma = (i_1\;i_r)(i_1\;i_{r-1})\cdots(i_1\;i_2)$より,$\varepsilon(\sigma) = (-1)^{r-1}$.
    \item[3.] (a)$r_1,\cdots,r_k$の最小公倍数.(b)$(-1)^{r_1+\cdots+r_k-k}$.
    \item[4.] 
    \item[5.(a)] $(1\;3\;6\;7\;2)(4\;5)$.
    \item[5.(b)] $(1\;3\;5\;6)(2\;4)$.
    \item[6.(a,b)] $[1,1,1,1]:e$.
    $[2,1,1]:(1\;2),(1\;3),(1\;4),(2\;3),(2\;4),(3\;4)$.
    $[2,2]:(1\;2)(3\;4),(1\;3)(2\;4),(1\;4)(2\;3)$.
    $[3,1]:(1\;2\;3),(1\;3\;2),(1\;2\;4),(1\;4\;2),(1\;3\;4),(1\;4\;3),(2\;3\;4),(2\;4\;3)$.\\
    $[4]:(1\;2\;3\;4),(1\;2\;4\;3),(1\;3\;2\;4),(1\;3\;4\;2),(1\;4\;2\;3),(1\;4\;3\;2)$.
    \item[6.(c)] $o(S_4)=24$より,部分群の位数としてあり得るのは$24$の約数のうち,共役類の元の数$1,6,3,8,6$をたかだか一つずつ使う和で表現できるもの.これに従って確かめると,$\Set{e},\Set{e,[2,2]},A_4=\Set{e,[2,2],[3,1]},S_4$が正規.
\end{itemize}

\subsection{置換表現,群の集合への作用}
\begin{itemize}
    \item[1.] $(ab) \cdot Hx = Hx b^{-1}a^{-1} = a \cdot (b \cdot H)$,$e \cdot Hx = Hx$より.
    \item[2.] $aH \mapsto Ha^{-1}$は$G$同型写像で,$G$の$G/H$における表現と$G \backslash H$における表現は同値になる(well-defined性などを確認すること).
    \item[3.] \miss 定理16を使う.$aH \in G/H$の安定部分群は$\Set{x \in G | xaH = aH}$で$xaH = aH \Leftrightarrow x \in aHa^{-1}$.したがって$K = aHa^{-1}$と表されるならば(つまり$H,K$が共役ならば)定理16より$G/H$と$G/K$における表現は同値になる.逆は難しい.$G$同型写像を$\varphi$として,$\varphi(K) = aH$とする.このとき$\varphi(xK) = xaH$で,$x \in K$のときに限り$\varphi(xK) = \varphi(K) = aH$.したがって$x \in K \Leftrightarrow aH = xaH \Leftrightarrow x \in aHa^{-1}$より,$K = aHa^{-1}$.
    \item[4.] $X$を推移的$G$集合とする.ある$x \in X$についての安定部分群を$H$とする.このとき$X$と$G/H$における表現は同値.$X$における表現が忠実なので,$G/H$における表現も忠実.これは定理17の系により$H$が単位群以外に$G$の正規部分群を含まないことと同値だが,$G$は可換群なので任意の部分群は正規.したがって$H$は単位群以外の部分群を含まない,つまり$H$は単位群である.これより$G/H$における表現は$G$の左正則表現であり,$X$における表現と同値.
    \item[5.] $H < G,o(H) = p$とする.$H$は真部分群をもたない巡回群である.また$(G:H) = m$で,$m!$は$p$で割り切れないので,$H$は単位群以外の$G$の正規部分群を含む.つまり$H$自身が$G$の正規部分群である.
    \item[6.] \miss まず次の補題を示す:$H<G$として,$N \triangleleft G, N \triangleleft H$とする.このとき$H/N \triangleleft G/N \Leftrightarrow H \triangleleft G$.(証明):$aNxN(aN)^{-1} = axa^{-1}N$より,$aNxN(aN)^{-1} \in H/N$が成り立つのは$axa^{-1} \in H$が成り立つとき,またそのときに限る.\\
    まず$n=1$のときは明らか.$n>1$で,$o(G) = p^n, o(H) = p^{n-1}$とすると$(G:H) = p$.$p!$は$p^n$で割り切れないので,$H$は単位群以外の$G$の正規部分群を含む.これは当然$H$の正規部分群でもある.これを$N$として$o(N) = p^m$とすると,$H/N,G/N$はそれぞれ位数$o(H/N) = p^{n-m-1}, o(G/N) = p^{n-m}$だから,帰納法の仮定により$H/N \triangleleft G/N$.したがって補題より$H \triangleleft G$.
\end{itemize}

\subsection{直積}
\begin{itemize}
    \item[1.(a)] $\R^* = \pm\R_>^*$.$\R^*$は可換群で,$\R_>^*$と$\Set{\pm 1}$はどちらも$\R^*$の部分群なので正規.$\R_>^* \cap \Set{\pm 1} = \Set{1}$なので,$\R^*$はこれらの直積に分解される.
    \item[1.(b)] $\C^*$の任意の元が$re^{i\theta}$と表されることから$\C^* = \R_>^*\T$.$\C^*$は可換群で,$\R_>^*$と$\T$はどちらも$\C$の部分群なので正規.$\R_>^* \cap \T = \{1\}$なので,$\C^*$はこれらの直積に分解される.
    \item[2.] $\varphi_1:G \to N_1$を$xy \mapsto x(x \in N_1, y \in N_2)$とすれば,$\varphi_1$は全射準同型で,$\Ker \varphi_1 = N_2$.したがって準同型定理より$G/N_2 \simeq N_1$.
    \item[3.] 明らか.
    \item[4.] $G$は位数$pq$の巡回群.真部分群の位数は$p,q$で,それぞれ$G_1,G_2$のみが対応する.
    \item[5.] $G \times G$の位数は$p^2$だから,真部分群の位数としてあり得るのは$p$.$a$を$G$の生成元とする.このとき$\langle(a,e)\rangle,\langle(a,a)\rangle,\langle(a,a^2)\rangle,\cdots,\langle(a,a^{p-1})\rangle,\langle(e,a)\rangle$は部分群になる.部分群は$p+1$個存在する.$\langle(a^n, a)\rangle$は,$\langle(a,a^m)\rangle$と同じになってしまうので重複に注意.
    \item[6.] 「$G$が$N_1,\cdots,N_n$の直積に分解される $\Leftrightarrow$ $N_i \triangleleft G$で(1),(2)をみたす」を示す.まず$(\Rightarrow)$を示す.(1)が成り立つのは明らか.まず$N_i \triangleleft G$を示す.任意の$x \in G$は$x = x_1x_2 \cdots x_n(x_i \in N_i)$と一意に表されるから,任意の$y \in N_i$について$xyx^{-1} = x_1 \cdots x_{i-1}x_{i+1} \cdots x_n x_i y x_i^{-1} x_n^{-1} \cdots x_{i+1}^{-1}x_{i-1}^{-1} \cdots x_1^{-1} = x_i y x_i^{-1} \in N_i$.したがって$N_i \triangleleft G$.次に(2)を示す.$x \in (N_1 \cdots N_{i-1}N_{i+1} \cdots N_n) \cap N_i$とすると,$x \in N_i$より$x = e \cdots x \cdots e$($i$番目が$x$),$x \in N_1 \cdots N_{i-1}N_{i+1} \cdots N_n$より$x = x_1 \cdots x_{i-1} x_{i+1} \cdots x_n$のように表されるが,一意性より両者は一致しなければならないので,$x = x_j = e$.したがって(2)も成り立つ.次に$(\Leftarrow)$を示す.(1)より$G=N_1 \cdots N_n$は明らかなので,$N_i,N_j$が可換なことと,表示が一意なことが言えればよい.まず可換であることは問題2.5.14と同様にしてわかる.一意性を示す.$x = x_1 \cdots x_n = y_1 \cdots y_n$とすると,$(x_1^{-1}y_1) \cdots (x_n^{-1}y_n) = e$より$(x_1^{-1}y_1) \cdots (x_{i-1}^{-1}y_{i-1})(x_{i+1}^{-1}y_{i+1}) \cdots (x_n^{-1}y_n) = x_iy_i^{-1}$.これより両辺は$e$に等しく,これが任意の$i$で成り立つので$x_i = y_i$,つまり表示は一意である.
\end{itemize}

\subsection{Sylowの定理}

\section{環と多項式}
\subsection{環とその例}
\begin{description}
    \item[非可換環の例1] 加法群$\Z^2$は可換群で,自己準同型写像全体は要素が整数の2次正方行列で表せる.つまり$\End(\Z^2) = M(2,\Z)$である.これは一般に非可換である.$\End(\Z^2) = M(2,\Z)$となることを示す.
    任意の$x \in \Z^2$は$m,n \in \Z$によって
    \begin{equation}
        x = m\mqty[1\\0] + n\mqty[0\\1]
    \end{equation}
    と表せる.
    \footnote{$\Z^2$だから基底を整数回足し合わせて任意の元を作れるが,$\R^2$ではそれができないことに注意.}
    したがって任意の自己準同型写像$f$による像は
    \begin{equation}
        f(x) = mf\left(\mqty[1\\0]\right) + nf\left(\mqty[0\\1]\right)
    \end{equation}
    のようになる.ここで
    \begin{equation}
        \mqty[f\left(\mqty[1\\0]\right) & f\left(\mqty[0\\1]\right)] = \mqty[\mqty[1\\0] & \mqty[0\\1]]\mqty[\xmat*{a}{2}{2}] \quad (a_{ij} \in \Z)
    \end{equation}
    のように基底が移り変わるとすると,右辺の行列を$A$として
    \begin{equation}
        f(x) = A\mqty[m\\n]
    \end{equation}
    と表せる.従って任意の自己準同型写像は$M(2,\Z)$の元で表せる.逆は明らかなので,$\End(\Z^2) = M(2,\Z)$が示された.
\end{description}

\begin{itemize}
    \item[1.] 単位元は$R$の単位元への定値写像.$M(S,R)$が可換なとき,任意の$f,g \in M(S,R)$について,
    \begin{equation}
        \forall x \in S, (fg)(x) = (gf)(x) \therefore f(x)g(x) = g(x)f(x)
    \end{equation}
    ここで$f(x),g(x)$は,$f,g$を動かすと$R$の元全体を取りうるので,$M(S,R)$が可換になるのは$R$が可換なときのみ.

    \item[2.] 明らか.
    \item[3.] 明らか.
    \item[4.] 前問の結果を使えばすぐわかる.
    \item[5.] 公式
    \begin{equation}
        \mqty(n+1\\k) = \mqty(n\\k) + \mqty(n\\k-1)
    \end{equation}
    を用いて,数学的帰納法によって示せる(可換環なので特に気にすることもない).
    \item[6.] 明らか.
    \item[7.] 任意の$x,y \in R$について
    \begin{align}
        (x + y)^2 &= (x + y)(x + y) = x^2 + xy + yx + y^2 \\
        x + y &= x + y + xy + yx \\
        xy &= -yx
    \end{align}
    特に$y = 1$としたら$x = -x$だから,$xy = -yx = yx$より,$R$は可換環.$n \in \Z$について$nx = (n \mod 2)x$となり,たしかにBooleらしい.
    \item [8.] 環になることを示す.積に関して,結合律・単位元$S$の存在はすぐわかる.分配法則に関しては,
    \begin{equation}
        A \cap (B \triangle C) = (A \cap B) \triangle (A \cap C)
    \end{equation}
    などがわかる(図を描くと良い).これより$P(S)$は環になって,$A \cap A = A$なのでBoole環である.
\end{itemize}

\subsection{整域,体}
整域の性質についての補足
\begin{itemize}
    \item 
    \begin{equation}
        ab = 1 \Rightarrow ba = 1
    \end{equation}
    証明:
    \begin{equation}
        ab = 1 \Rightarrow b = bab \quad \therefore (1 - ba)b = 0 \quad \therefore 1 = ba
    \end{equation}
    \item $a \in R(a \neq 0)$を固定した写像$x \mapsto ax$は単射である:
    \begin{equation}
        ax = ay \Leftrightarrow a(x - y) = 0 \quad \therefore x = y
    \end{equation}
\end{itemize}
問題解答.
\begin{itemize}
    \item[1.] $a \in R$を単元とする.$a$が零因子でもあると仮定すると,ある$b \in R, b \neq 0$が存在して,$ab = 0$.この両辺に$a^{-1}$を左からかければ$b = 0$となり,矛盾.したがって$a$は零因子ではない.
    \item[2.] $f \in M(S,R)$で,ある$c \in S$について$f(c) = 0$となるとき,
    \begin{equation}
        g(x) = \begin{cases}
            0 & (x \neq c) \\
            1 & (x = c)
        \end{cases}
    \end{equation}
    とすれば$fg = 0$となる.したがって$f$は零因子である.一方全ての$x \in S$で$f(x) \neq 0$ならば,$R$が体だから$f(x)$の逆元が存在するので,$f^{-1}(x) = f(x)^{-1}$とすれば$f^{-1}f = ff^{-1} = 1$.
    \item[3.] 例えば$f(m,n) = (m+n, m+n)$,$g(m,n) = (m+n, -(m+n))$とすれば,$fg = 0$になる(3.1節で与えた例のように行列で考えるとよい).
    \item[4.] \textcolor{red}{\#}$a$が左右どちらの零因子でないとする.このとき写像$x \mapsto ax$は単射である:$ax = ay \Leftrightarrow a(x-y)=0$のとき,$a$は零因子でないから$x = y$.同様に$x \mapsto xa$も単射.$R$が有限集合だからどちらも全射でもあるので,逆写像,つまり逆元が存在する.
    \item[5.] $\mathrm{(i)} \Rightarrow \mathrm{(ii)}$:$b \neq b',ab'=1$とすると$1 = ab = ab' \Leftrightarrow a(b - b') = 0$より,$a$は左零因子.$\mathrm{(ii)} \Rightarrow \mathrm{(i)}$:単元だとすると矛盾(1でやった).$\mathrm{(iii)} \Rightarrow \mathrm{(i)}$:$a$が単元でないとすると$ba \neq 1$なので,\textcolor{red}{\#}$ba = 1 + u(u \neq 0)$と表せる.この両辺に$a$を左からかけると$aba = a + au$となるが,$aba = a$なので$au = 0$.したがって,$b' = b + u$ととれば$ab' = a(b + u) = 1$.
    \item[6.] 零元は$0$.逆元は$-(a+bi)$.単位元は$1$.$a+bi$に乗法の逆元が存在すれば,それは$(a-bi)/(a^2+b^2)$の形.整数範囲ならば$a=0,b=\pm 1$,$a=\pm 1,b=0$のときのみ逆元が存在する.つまり単元は$\pm 1, \pm i$のみ.
    \item[7.] たとえば$\sqrt{2}$は単元ではない.
    \item[8.]  
\end{itemize}

\subsection{イデアルと商環}
\begin{itemize}
    \item[1〜4.] 容易に確認できる.
    \item[5.] \miss
    \item[6〜8.] 容易に確認できる.
    \item[9,10.] $x^m = 0,y^n = 0(m \leq n)$とする.このとき$(x + y)^{2n} = 0$,$\forall r \in R, (rx)^n = r^nx^n = 0$より$N$はイデアル.$R/N$の元で$\bar{a}$をべき零元とすると,$\bar{a}^n \Leftrightarrow a^n \equiv 0(\Mod N) \Leftrightarrow a \in N$より,$\bar{a} = \bar{0}$.
    \item[11.] 容易に確認できる.
\end{itemize}

\subsection{$\Z$の商環}
\begin{itemize}
    \item[1.] $\Z_p$は真部分群を持たないので,取りうるイデアルは$\{0\},\Z_p$のいずれか.したがって$\Z_p$は体である.
    \item[2.] $a$を$G$の生成元とする.任意の$f \in \Aut(G)$について,$f(a^m) = f(a)^m$だから,$f(a)$は$G$の生成元になっていなければならない.2章8節問題3により,これは$(k,n)=1$なる自然数$k$によって$f_k(a)=a^k$と表されることを意味する.したがって$\Aut(G)$は$n$と互いに素な$n$未満の自然数$k$によって$f_k$と表される自己同型写像全体で位数は$\varphi(n)$.これは法$n$に関する$\Z$の既約剰余類群と同型.
    \item[3.] \miss $m = a^n-1$とする.$(a,m) = 1$だから$\bar{a}$は法を$m$とする既約剰余類群に含まれる.また$a^n \equiv 1(\Mod m)$だから$\bar{a}^n = \bar{1}$で,$a^n = m+1$だから$n$が$\bar{a}^k = 1$をみたす$k$のうちで最小である.したがって$\bar{a}$を生成元とする巡回群の位数は$n$だから,$n|\varphi(m)$.
    \item[4.] \miss 問題は,法を$n$とする既約剰余類群から位数$p$の部分群を取り出せるか,に言い換えられる.素数位数の部分群は巡回群である.したがってその生成元を$\bar{a}$とすれば$\bar{a}^p = 1$となる.そしてこれはSylowの定理により肯定される.
\end{itemize}
\end{document}