\documentclass[a4paper,10pt,uplatex]{jsarticle}


% 数式
\usepackage{amsmath,amssymb,amsthm,bm}
\usepackage{mathtools}
\mathtoolsset{showonlyrefs}
\usepackage{physics}
\usepackage{siunitx}
\usepackage[thicklines]{cancel}
\usepackage{comment}
\usepackage{graphicx}
\usepackage[dvipdfmx]{color}
\usepackage{here}
\usepackage{tikz}
\usepackage{enumitem}
% \setlist{nolistsep}

\newcommand{\Set}[1]{\{#1\}}
\newcommand{\miss}{\textcolor{red}{\#}}
\newcommand{\N}{\mathbb{N}}
\newcommand{\R}{\mathbb{R}}
\newcommand{\C}{\mathbb{C}}
\newcommand{\Z}{\mathbb{Z}}
\newcommand{\Q}{\mathbb{Q}}
\newcommand{\T}{\mathbb{T}}
\newcommand{\inv}[1]{{(#1)}^{-1}}
\newcommand{\Mod}{\text{mod}\;}
\newcommand{\End}{\text{End}}
\newcommand{\Ker}{\text{Ker}\;}
\newcommand{\Aut}{\text{Aut}}
\newcommand{\transpose}[1]{^t\! #1}

\numberwithin{equation}{section}

\newtheoremstyle{mystyle}%
    {5pt}%
    {5pt}%
    {}%
    {\parindent}%
    {\bfseries}%
    {:}%
    { }%
    {}%
\theoremstyle{mystyle}
\newtheorem{dfn}{定義}[section]
\newtheorem{thm}{定理}[section]
\newtheorem{lem}[thm]{補題}
\newtheorem{cor}[thm]{系}
\renewcommand{\proofname}{\textbf{証明}}

% 数式の上下のスペースの変更
\AtBeginDocument{
  \abovedisplayskip     =0.5\abovedisplayskip
  \abovedisplayshortskip=0.5\abovedisplayshortskip
  \belowdisplayskip     =0.5\belowdisplayskip
  \belowdisplayshortskip=0.5\belowdisplayshortskip
}

\begin{document}

\title{池田「テンソル代数と表現論」}
\author{}
\date{\today}
\maketitle

課題の解答例.
\section{広義固有区間}
\begin{itemize}
    \item[1.1] $A^m = E$より,$f(t) = t^m - 1$は$f(A) = 0$をみたす.
    \begin{equation}
        f(t) = \prod_{k=0}^{m-1}(t-e^{2\pi i k/m})
    \end{equation}
    である.最小多項式は$f$を割り切るので,$f$の右辺のどの項も高々1回しか現れない,つまり重根を持たない.したがって定理1.2.6より$A$は対角化できる.
\end{itemize}

\section{ジョルダン標準形}

\section{行列の指数関数とその応用}

\section{テンソル代数}
\begin{itemize}
    \item [4.1] $\transpose{f}$の表現行列を$B$とする.
    \begin{equation}
        \langle \transpose{f}(\psi_i),v_j \rangle = \langle \sum_{k} b_{ki}\phi_k, v_j \rangle = \sum_{k} b_{ki}\delta_{jk} = b_{ji}
    \end{equation}
    と
    \begin{equation}
        \langle \transpose{f}(\psi_i),v_j \rangle = \psi_i(f(v_j)) = \psi_i \left(\sum_{k} a_{kj} w_k\right) = \sum_{k} a_{kj}\delta_{ik} = a_{ij}
    \end{equation}
    より,$b_{ji} = a_{ij}$.したがって$B = \transpose{A}$.

    \item[4.2(1)] $\psi \in  W^*$の定義域を$V$に拡張すれば$V^*$の元になる.それには$\psi(v_{r+1}), \cdots, \psi(v_n)$を設定して,線型に拡張すればよい.したがって$\Phi : W^* \to V^*$は全射.
    \item[4.2(2)] $\bar{\Psi}$を$\phi + W^{\perp} \mapsto \phi|_W$と定義すればwell-definedな線型写像になる.
    \item[4.2(3)] $\dim (V^*/W^{\perp}) = n - \dim W^{\perp} = n - (n - r) = r = \dim W^*$より$\bar{\Psi}$は全射.(2)より単射でもあるから,線型同型.
\end{itemize}

\end{document}